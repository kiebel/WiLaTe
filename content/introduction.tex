% vim:tw=78:ai:bg=light:set spell:spelllang=de:set nu
%%%%%%%%%%%%%%%%%%%%%%%%%%%%%%%%%%%%%%%%%%%%%%%%%%%%%%%%%%%%%%%%%%%%%%%%%%%%
%%% Motivation
%%%%%%%%%%%%%%%%%%%%%%%%%%%%%%%%%%%%%%%%%%%%%%%%%%%%%%%%%%%%%%%%%%%%%%%%%%%%
\chapter{Einleitung}

In \cite{manual-1999} werden hilfreiche Tipps für die Erstellung einer \ac{MA} gegeben.

\section{Motivation}
\todoComment{RevisorA}{Motivation: aktueller Stand der Technik und Beantwortung
der Frage warum diese Arbeit interessant/wichtig ist?}

\section{Aufgabenstellung}
\ldots\thought{Ein Gedanke}

Laut Gödel\doCite{Zitat?} kann ein Axiomensystem nie vollständig bewiesen werden, Gödelscher Unvollständigkeitssatz\lookup{Das muss nochmal nachgeschlagen werden}

\section{Gliederung der Arbeit}
\todo[inline,caption={Gliederung}]{Hier is noch was zu tun}

\begin{itemize}
	\item \todoNote{RevisorA}{freundlicher Hinweis}{irgendwas} 
	\item \todoError{RevisorA}{Das ist falsch}{f\"ur ein rechtwinkliges Dreieck gilt: $a+b=c$} 
	\item \todoRewrite{RevisorA}{Unlesbar nochmal neu schreiben}{Det isch abscholuder Unnschinn} 
\end{itemize}
%%%%%%%%%%%%%%%%%%%%%%%%%%%%%%%%%%%%%%%%%%%%%%%%%%%%%%%%%%%%%%%%%%%%%%%%%%%%
\cleardoublepage
